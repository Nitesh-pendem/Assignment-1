\let\negmedspace\undefined
\let\negthickspace\undefined
\documentclass[journal]{IEEEtran}
\usepackage[a5paper, margin=10mm, onecolumn]{geometry}
%\usepackage{lmodern} % Ensure lmodern is loaded for pdflatex
\usepackage{tfrupee} % Include tfrupee package

\setlength{\headheight}{1cm} % Set the height of the header box
\setlength{\headsep}{0mm}     % Set the distance between the header box and the top of the text

\usepackage{gvv-book}
\usepackage{gvv}
\usepackage{cite}
\usepackage{amsmath,amssymb,amsfonts,amsthm}
\usepackage{algorithmic}
\usepackage{graphicx}
\usepackage{textcomp}
\usepackage{xcolor}
\usepackage{txfonts}
\usepackage{listings}
\usepackage{enumitem}
\usepackage{mathtools}
\usepackage{gensymb}
\usepackage{comment}
\usepackage[breaklinks=true]{hyperref}
\usepackage{tkz-euclide} 
\usepackage{listings}
% \usepackage{gvv}                                        
\def\inputGnumericTable{}                                 
\usepackage[latin1]{inputenc}                                
\usepackage{color}                                            
\usepackage{array}                                            
\usepackage{longtable}                                       
\usepackage{calc}                                             
\usepackage{multirow}                                         
\usepackage{hhline}                                           
\usepackage{ifthen}                                           
\usepackage{lscape}
\begin{document}

\bibliographystyle{IEEEtran}
\vspace{3cm}


\title{1-1.4-4}
\author{AI24BTECH11026 - Pendem nitesh sri satya$^{*}$% <-this % stops a space
}
\maketitle
\begin{enumerate}
    \item Find the coordinates of the point which divides the line segment joining the points $(4,-3)$ and $(8,5)$ in the ratio $3:1$ internally\\
\solution

\begin{table}[h!]    
  \centering
  \begin{tabular}[12pt]{ |c| c|}
    \hline
    \textbf{Variable} & \textbf{Description}\\ 
    \hline
    $A$ & position vector of point (4, -3) \\
    \hline 
    $B$ & position vector of point (8, 5)\\
    \hline
    $P$ & position vector of point whhich divides points A and B in the ratio 3:1\\
    \hline   
    \end{tabular}

  \caption{Variables Used}
  \label{tab10.5.3.9.1}
\end{table}
Let the position vectors of the points \((4, -3)\) and \((8, 5)\) be represented as A and B respectively. Therefore, we have:
\begin{align}
A = 4i - 3j
\end{align}
\begin{align}
B = 8i + 5j
\end{align}

Let the position vector of the point P that divides the line segment AB in the ratio 3:1 internally be P.

Using the section formula in vector form, the position vector P is given by:
\begin{align}
P = \frac{m{B} + n{A}}{m+n}
\end{align}
where m = 3 and n = 1

Substituting the values, we get:
\begin{align}
P = \frac{3(8i + 5j) + 1(4i - 3j)}{3+1}
\end{align}
\begin{align}
P = \frac{(24i + 15j) + (4i - 3j)}{4}
\end{align}
\begin{align}
P = \frac{(24i + 4i) + (15j - 3j)}{4}
\end{align}
\begin{align}
P = \frac{28i + 12j}{4}
\end{align}
\begin{align}
P = 7i + 3j
\end{align}

Therefore, the coordinates of the point are (7, 3).
\end{enumerate}
\end{document}

